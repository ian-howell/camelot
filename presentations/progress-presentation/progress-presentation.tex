\documentclass{beamer}

\newcommand{\versionNumber}{1.1}
\newcommand{\lastEditAuthor}{Illya Starikov}

% __ Draft   __ Proposed  __ Validated  __ Approved
\newcommand{\documentStatus}{Proposed}


\EdefRot{13}\joke{maybe if we git gud, we'll use ROT26.}
\newcounter{tools}

\title{Camelot Progress Presentation}
\subtitle{Software Engineering}
\author{Ian Howell, Hunter Mathews, Illya Starikov, William Thurman, Zachary Wileman (\textit{Server Team \#1})}
\date{March 24\textsuperscript{th}, 2017}
\institute{Missouri University of Science and Technology}

\begin{document}
\maketitle


% Zachary Wileman
% TODO: Create this slide to give an overall view of how the project works
\begin{frame}
    \frametitle{Camelot}

    Camelot is an asynchronous \textit{Python3} server, using \textit{PostgreSQL} for the database.
\end{frame}


% Zachary Wileman
\begin{frame}
    \frametitle{Camelot Inspiration}

    \begin{figure}
        \centering
        \includegraphics[width=\textwidth]{images/camelot}
        \caption{"On second thought, let's not go to Camelot. 'Tis a silly place." — Arthur, Monty Python}
        \label{fig:camelot}
    \end{figure}
\end{frame}


% William Thurman
\begin{frame}
    \frametitle{What Has Worked Well}
    Below are some of the things has worked well for us up to this point.

    \begin{description}[<+->]
        \item[Teamwork] Task delegation and team work has made us able to accomplish more, in less time.
        \item[Teammates] Have a wide array of complementing skills and backgrounds --- we're pretty much the lowkey OP group.
        \item[Tools] By picking some of the best tools (Github, \LaTeX{}, PostgreSQL), it has made for a much better workflow.
    \end{description}
\end{frame}


% Illya Starikov
\begin{frame}
    \frametitle{Past Changes}
    Below are past decisions that if we could change, we would. \pause

    % This is intentionally an MLG frog
    \centering
    \animategraphics[loop, controls, width=.75\linewidth]{12}{images/frog-}{0}{39}
\end{frame}


% Illya Starikov
\begin{frame}
    \frametitle{Tools \presentcount{tools}}
    A summary of some of the tools we're using/enjoying.

    \begin{itemize}
        \item Written in Python3.
        \begin{itemize}
            \item Documentation is written in Doxygen.
        \end{itemize}

        \item All development is on Github.
        \begin{itemize}
            \item Git for version history.
            \item Github issues/milestones for task management.
            \item Contributions to make sure even workload.
        \end{itemize}

        \item All documentation/presentation is written in \LaTeX{}.
        \item Using ProgreSQL for the database.
        \item Using Discord for team/client chat.
    \end{itemize}
\end{frame}


% Illya Starikov
\begin{frame}
    \frametitle{Tools \presentcount{tools}}

    \LaTeX{} makes documentation look professional.

    \centering
    \frame{\includegraphics[page=1, scale=.25]{../../documentation/requirements.pdf}}
\end{frame}


% Ian Howell
\begin{frame}[fragile,c]
    \frametitle{Tools \presentcount{tools}}

    Python enables easy server setup.

    \begin{lstlisting}[style=cpython]
    def main():
        server = ThreadedTCPServer(("0.0.0.0", 9009), ThreadedTCPRequestHandler)
        ip, port = server.server_address
        server.socket.listen(10)

        server_thread = threading.Thread(target=server.serve_forever)
        server_thread.dameon = True
        server_thread.start()
        print("Server loop running in thread:", server_thread.name)

        try:
            # Loop forever
            while True:
                pass
        except KeyboardInterrupt:
            print("Cleaning up server....")
            server.shutdown()
            server.server_close()
            print("Done! Goodbye")

    \end{lstlisting}
\end{frame}


% Ian Howell
\hugeslide{Server Demo}


% Hunter Matthews
\begin{frame}
    \frametitle{Tools \presentcount{tools}}

    Github keeps track of all contributions from team members.

    \centering
    \frame{\includegraphics[width=.60\textwidth]{images/github}}
\end{frame}


% William Thurman
\begin{frame}
    \frametitle{Challenges}
    We have encountered three big challenges so far.

    \begin{enumerate}[<+->]
        \item Communication between the client and server teams.
        \item Direction and vision of the product.
        \item Learning curve of tools, teammates, and project.
    \end{enumerate}

    \begin{block}{Description \& Solution}
        \only<1>{Even though we have an established system has been established between the teams (Discord), there is little to no talk to client and server teams. \textit{Solution: Communicate more..?}}
        \only<2>{Although there is an established, core vision (IRC clone), working out the detail has proven to be difficult. \textit{Solution: Break it down to basic principles and work with teammates.}}
        \only<3>{Because we're relatively new to each other, and are unaware of each other's skill sets, assigning tasks becomes tricky. Also learning how to implement new things can be difficult. \textit{Always be sure teammates are comfortable with the tasks they're given.}}
    \end{block}
\end{frame}


% Illya Starikov
\begin{frame}
    \frametitle{Extras}

    \begin{itemize}
        \item Encryption.
        \begin{itemize}
            \item Still debating if we'll be using ROT13 (\joke) or an in-house, post-quantum cryptographic hash function using a multivariate-quadratic public-key signature system.
        \end{itemize}

        \item Open source \pause at the end of the semester.
        \item Full documentation guide courtesy of Doxygen.
    \end{itemize}
\end{frame}


% Hunter Mathews
\begin{frame}
    \frametitle{Current and Future Plans}

    \begin{enumerate}
        \item Consolidate the client and server teams to get unified protocol and systems specification document.
        \item Finish coding the actual chatroom.
        \begin{itemize}
            \item Server
            \item Database
            \item Script to automate a chatroom environment
        \end{itemize}

        \item Finish documenting the chatroom (possibly API guide).

        \pause
        \item $\cdots$
        \item Profit
    \end{enumerate}
\end{frame}


% Illya Starikov
\begin{frame}
    \frametitle{In Closing}

    Spam Illya with all questions, comments, and insults.
    \begin{description}
        \item[\faGithub]  \href{https://github.com/IllyaStarikov}{\nolinkurl{@IllyaStarikov}}
        \item[\faComment] \href{mailto:starikov@mst.edu}{\nolinkurl{starikov@mst.com}}
    \end{description}

    Special thanks to our awesome team.
    \begin{description}
        \item[\faUser] Ian Howell
        \item[\faUser] Hunter Mathews
        \item[\faUser] Illya Starikov
        \item[\faUser] William Thurman
        \item[\faUser] Zachary Wileman
    \end{description}
\end{frame}

\end{document}
