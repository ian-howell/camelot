\chapter{Introduction}\label{introduction}
Camelot is JSON-based server written exclusively in \href{https://www.python.org/about/}{Python 3}. The purpose of Camelot is to provide an IRC-esque chatroom on the server, leaving implementation details for the client up open to interpretation.

\section{Purpose}\label{purpose}
This document is intended to guide development of Camelot. Not only will it provide the requirements (i.e.~system requirements, user requirements, and interface requirements), but it will provide a decent outline for what the server entails (i.e.~functions, constraints, dependencies, etc.). Understanding the system requirements will give a more clear and concise understanding of the \acrfull{apis} and their purpose.

\section{How to Use This Document}\label{how-to-use-this-document}
Because this document will be reviewed by various different skill sets, this section will break down which parts should be reviewed by which type of reader.

\subsection{Types of Reader}\label{types-of-reader}
This document is going to be aimed at two different type of people: developers and end-users. The first set, client/server side developers, will be the \gls{python}3 Programmers, server-side programmers, \acrfull{it} personal, and others that might have communication needs. These people are going to be the ones who wish to build upon or plan to use the code commercially. The other set will be the end-users. These people will be running the server in order to hold a chat session(s).

\subsection{Technical Background
Required}\label{technical-background-required} Programming competency \emph{is imperative} to understanding the documentation. Programming methodology, jargon, and general concepts will be assumed in subsequent sections. Server side programming is recommended, but not required.

The document will stay to a high level server design. For the purpose of this document, source code will not appear. Specific tools/programming languages capability is not required.

\subsection{Overview Sections}\label{overview-sections}
To get a a general understanding of the document, this following sections and subsections should be (chronologically) read:

\begin{enumerate}
    \item User Characteristics (Section \ref{user-characteristics})
    \item Assumptions and Dependencies (Section \ref{assumptions-and-dependencies})
    \item Specific Requirements (Section \ref{specific-requirements})
\end{enumerate}

\subsection{Reader-Specific Sections}\label{reader-specific-sections}

Types of readers are described as follows.

\begin{itemize}
    \item Server-side Developers: This would entail anyone interested in working on this particularly instance of the server. All of this document should be read.
    \item Client-side Developers: This would entail all who would want to get involved in creating a client. The following sections should be read:
    \begin{enumerate}
        \item Description (Section \ref{description})
        \item Specific Requirements (Section \ref{specific-requirements})
    \end{enumerate}

    \item End Users: This would entail anyone that is generally interested, but has no intentions of truly using it standalone.
    \begin{enumerate}
      \item Product Perspective (Section \ref{product-perspective})
      \item Product Functions (Section \ref{product-functions})
      \item User Characteristics (Section \ref{user-characteristics})
    \end{enumerate}
\end{itemize}

\section{Scope of the Product}\label{scope-of-the-product}
Truthfully, the intention of this project is to get an A in software engineerings. But I actually have to put something here, so.

There is an expectation at the end of the development lifecycle to open source this server. After open sourcing, the server team hopes that Camelot will serve as a model for server-side programming. Because the \gls{json}-based framework would be familiar to people, the team hopes for a some market adoption. Overall, this would be a good model for how a simple server would be maintained.

\section{Business Case for the Product}\label{business-case-for-the-product}
There is a real pandemic: there are not enough chat applications. Sure, there's
\href{https://www.messenger.com}{Messenger},
\href{https://slack.com/}{Slack},
\href{https://www.skype.com/en/}{Skype},
\href{https://www.viber.com/en/}{Viber},
\href{https://web.wechat.com}{WeChat},
\href{https://www.whatsapp.com}{WhatsApp},
\href{https://line.me/en/}{Line},
\href{https://web.groupme.com}{GroupMe},
\href{https://www.snapchat.com}{Snapchat},
\href{http://voxer.com}{Voxer},
but they are garbage. The market needs a great server to tackle on these giants, and Camelot will do that.

\section{Overview of the Requirements
Document}\label{overview-of-the-requirements-document}
\begin{enumerate}
    \item Raspberry Pi 3 needed to host the server
    \item Knowledge of Python to program the server
    \item JSON based framework
\end{enumerate}
